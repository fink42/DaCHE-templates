%%%%%%%%%%%%%%%%%%%%%%%%%%%%%%%%%%%%%%%%%%%%%%%%%%%%%%%%%%%%%%%%%%%%%%%%%%%%%%%%
%%%                                Preamble                                  %%%
%%%                             Version: 1.00                                %%%
%%% 																		 %%%
%%% 					DaCHE LaTeX template UK 4:3 						 %%%
%%%                       Author: Nicolai Simonsen                           %%%
%%%%%%%%%%%%%%%%%%%%%%%%%%%%%%%%%%%%%%%%%%%%%%%%%%%%%%%%%%%%%%%%%%%%%%%%%%%%%%%%
% Documentclass
\documentclass[aspectratio=43]{beamer} % Add [handout] to disable overlays

% Packages
\usepackage{amsmath}
\usepackage{amssymb}
\usepackage{array}
\usepackage{cancel}
\usepackage{caption}
\usepackage{dcolumn}
\usepackage{float}
\usepackage{graphicx,calc}
\usepackage{helvet}
\usepackage{hyperref}
\urlstyle{same}
\usepackage[utf8]{inputenc}
\usepackage{microtype}
\usepackage{natbib}
\usepackage{pifont}
\usepackage{pgfpages}
\usepackage{subcaption}
\usepackage{textpos}
\usepackage{threeparttable}
\usepackage{verbatim}
\usepackage{xcolor}
\usepackage{xfrac}
% Tikz
	\usepackage{tikz}
	\usetikzlibrary{arrows.meta}
	\usetikzlibrary{decorations.pathreplacing}
	\usetikzlibrary{shapes.multipart}

% Notes (Used for enabling and disabling notes on second screen)
	%\setbeameroption{show notes on second screen}
	%\setbeameroption{show only notes}

% Changes for DaCHE template
\input{./template/dache_template_uk_standard}

% Start of the presentation ----------------------------------------------------
%Information to be included in the title page:
\title{DaCHE templates}
\subtitle{PowerPoint and \LaTeX}
\author{\small Dorte Gyrd-Hansen}
\institute{%DaCHE - Danish Centre for Health Economics \\
	%University of Southern Denmark\\
	{\scriptsize } Strategy workshop 2019, Odense, Denmark\\
	Joint work with Nicolai Fink Simonsen and Anne Sophie Oxholm
}
\date{\today}

\begin{document}

	%%%%% Title page %%%%%
	{
	\metroset{background=dark}
	\begin{frame}
		\titlepage
		
		\note{
			\begin{itemize}
				\item This is a note
			\end{itemize}
		}

	\end{frame}
	}

\section{First section}


	%% New Frame %%
	\begin{frame}{Aim}
	
		\begin{itemize}
			\item Strengthen the DaCHE brand 
			\item Reduce workload for DaCHE employees
			\item Improve DaCHE employees presentations
		\end{itemize}
		
		\note{
			\begin{itemize}
				\item My own notes
		\end{itemize}}
	
	\end{frame}


	%% New Frame %%
	\begin{frame}{Templates}
		\begin{itemize}
			\item Access the templates
			\begin{itemize}
				\item DaCHE's PF-drive
				\item DaCHE's sharepoint
			\end{itemize}
			\item Choose different formats
			\begin{itemize}
				\item Danish (DK)/English (UK)
				\item Wide/standard frame
				\item PowerPoint/LaTeX
			\end{itemize}
		\end{itemize}
	\end{frame}


%% Last Frame ------------------------------------------------------------------
{
	\metroset{background=dark}
	
	\begin{frame}
		\begin{minipage}{\linewidth}
			\LARGE \textbf{Thank you for your attention!}
		\end{minipage}
		
		{
			\setlength{\fboxrule}{0pt}
			\fboxsep=0pt
			\noindent\fbox{
				\begin{minipage}{.65\linewidth}
					\rule[-.9cm]{10mm}{1pt}\\
					{\scriptsize Dorte Gyrd-Hansen}\\
					\scriptsize{\inlinegraphics{pictures/envelope-regular.png} \quad dgh@sdu.dk}\\[1mm]
					\scriptsize{\inlinegraphics{pictures/twitter-brands.png} \quad \href{http://twitter.com/DorteGyrd}{@DorteGyrd}}\\[1mm]
					\scriptsize{\inlinegraphics{pictures/globe-solid.png} \quad \url{https://portal.findresearcher.sdu.dk/en/persons/dgh}}
					
					\rule[-.8cm]{10mm}{1pt}\\
					{\scriptsize Danish Centre for Health Economics - DaCHE}\\
					\scriptsize{\inlinegraphics{pictures/twitter-brands.png} \quad \href{http://twitter.com/DaCHE_SDU}{@DaCHE\_SDU}}\\[1mm]
					\scriptsize{\inlinegraphics{pictures/globe-solid.png} \quad \href{http://dache.dk}{dache.dk}}\\[1mm]
					
					% SDU lower left
					\begin{textblock*}{100mm}(-0.88cm,1.32cm)
						\includegraphics[height=.98cm]{pictures/SDU-DaCHE-uk-2.png}
					\end{textblock*}
				\end{minipage}\quad
				
				\begin{minipage}{.3\linewidth}
					\begin{figure}
						\centering
						\includegraphics[width=0.98\linewidth]{pictures/DaCHE-profilbillede-round}
					\end{figure}
					
				\end{minipage}
			}
		}
		
		
		
	\end{frame}
}






%%%%%%%%References%%%%%%%%%
\bibliographystyle{apalike}
\bibliography{bibliography/bibliography.bib} % Path to bibliography file



\end{document}
